\section{Introduction}
\setlength{\parindent}{5ex}
Dans le cadre du projet informatique de S2 à EPITA, nous devons réaliser un jeu ou une application pour mettre en pratique les différentes connaissances apprises en TP et en cours.
Le projet de notre studio gameHUB sera donc un jeu vidéo du nom de Nyctalopia. Ce jeu sera créé à l’aide du moteur de jeu Unity Engine 3D et codé avec le langage C\#.

\setlength{\parindent}{5ex} 
Le jeu sera un survival horror mélangeant des styles de jeux indépendants aux styles de grands noms de l'horreur tel que Outlast.
Le joueur sera plongé dans la pénombre et aura un champ de vision réduit afin d'augmenter l'effet de surprise et la difficulté pour s'orienter, car il sera perdu sur une carte. Il sera constamment sous la menace d'une ou des entités présentes qui lui nuiront grâce à l'intelligence artificielle développée. Enfin nous ferons attention à ne pas désorienter l'utilisateur mais plutôt de lui proposer une immersion pour qu'il est envie de finir le jeu.

\setlength{\parindent}{5ex}
Ce projet nous permettra d’évoluer dans notre scolarité à EPITA et d’acquérir
un bon nombre de compétences en programmation qui nous seront nécessaires
dans nos carrières professionnelles. Le but final sera d'avoir un jeu complet avec un scénario et un gameplay agréable tout en étant imprévisible pour permettre à l'utilisateur de ne pas s'ennuyer.

Ce cahier des charges présente les choix du groupe sur la nature du projet
avec leurs origines, les différents objectifs ainsi que les besoins que ces choix
créent et comment le groupe compte y répondre. Dans ce document, le projet
est découpé par fonctionnalités et chaque partie est expliquée avec sa répartition
des tâches entre les étudiants du groupe. Le planning prévu pour l’avancement
des différents objectifs en considération est également présent.

\vfill
\noindent\makebox[\linewidth]{\rule{.8\paperwidth}{.6pt}}\\[0.2cm]
EPITA Toulouse - Projet S2 - 2021/2022 \hfill Nyctalopia - gameHUB
\noindent\makebox[\linewidth]{\rule{.8\paperwidth}{.6pt}}

\newpage
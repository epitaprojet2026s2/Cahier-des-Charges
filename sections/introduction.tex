\section{Introduction}
Dans le cadre du projet informatique de S2 à EPITA, les étudiants du groupe gameHUB doivent réaliser un jeu pour utiliser les différentes connaissances apprises en TP et en cours. 
Le projet du groupe a pour nom Nyctalopia (traduction anglaise du mot nyctalopie, qui est la faculté de voir dans la pénombre (vision crépusculaire ou nocturne), ou l'incapacité à bien voir dans un éclairage diurne, ayant pour conséquence de mieux voir dans la pénombre.) Ce projet sera créé à l’aide du moteur de jeu Unity Engine 3D et codé avec le langage C\#.

Le jeu sera donc du style "horreur" mélangeant des styles de jeux indie aux styles de grands AAA tel que Outlast.
Le joueur devra réa

Ce cahier des charges présente les choix du groupe sur la nature du projet
avec leurs origines, les différents objectifs ainsi que les besoins que ces choix
créent et comment le groupe compte y répondre. Dans ce document, le projet
est découpé par fonctionnalités et chaque partie est expliquée avec sa répartition
des tâches entre les étudiants du groupe. Le planning prévu pour l’avancement
des différents objectifs en considération est également présent.


Ce projet nous permettra d’évoluer dans notre scolarité à EPITA et d’acquérir
un bon nombre de compétences en programmation qui nous seront nécessaires
dans nos carrières professionnelles.

\vfill
\noindent\makebox[\linewidth]{\rule{.8\paperwidth}{.6pt}}\\[0.2cm]
EPITA Toulouse - Projet S2 - 2021/2022 \hfill Nyctalopia - gameHUB
\noindent\makebox[\linewidth]{\rule{.8\paperwidth}{.6pt}}

\newpage